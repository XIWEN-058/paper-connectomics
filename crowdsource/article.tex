\documentclass[wcp]{jmlr}

\usepackage{enumerate}
\usepackage{bbold}
\usepackage{booktabs}
\usepackage{natbib}
\usepackage[utf8]{inputenc}
\usepackage[super]{nth}
\setlength{\textfloatsep}{15pt}

\jmlryear{2014}
\jmlrworkshop{Neural Connectomics Workshop}

\title{Simple connectome inference from partial correlation statistics in calcium imaging}

  \author{\Name{Antonio Sutera},
   \Name{Arnaud Joly},
   \Name{Vincent François-Lavet}, \Email{a.sutera@ulg.ac.be}\\
   \Name{Zixiao Aaron Qiu},
   \Name{Gilles Louppe},
   \Name{Damien Ernst}\and\Name{Pierre Geurts}
    \\
   \addr Department of EE and CS \& GIGA-R, University of Li\`ege, Belgium}

\begin{document}

\section{A large variety of performing methods}

\emph{
In the Results section "A large variety of performing methods", we intend to
assemble together various layman presentations of different approaches. If you
could provide a short popular-science style description of your approach (and
specifically of what you consider their main characterizing ingredients,
INVERSE COVARIANCE for you) this would be very helpful for me.  Here is not
really an issue of being precise, but of commenting substantial elements to a
public of... ignorant biologists! What's the logic of making a certain step?
How does the method work in a nutshell, without using equations but expressed
in plain words, etc. ? Then, based on all the texts I will collect from you, I
would build the section and let you edit it to correct mistakes or add
precisions.\\
}

<<<<<<< HEAD
The inference problem reduced to its simpliest shape is the identification of
neuron pairs that are directly connected in the network. Intuitively, such
pairs should be related somehow and in particular their activities have to be
similar since their are dependent on the connectivity network. Therefore a way
to identify them is to measure, or at least estimate, how closely related two
neurons are regarding their activities. The most common measure for that
purpose is the (Pearson) correlation.  However, correlation also measures the
degree of association between neurons that are connected through others and so 
connected indirectly. It means all related neurons are found but direct pairs
are indinstinguishable from indirect ones.

A variant of this classic approach is the partial correlation which measures
the degree of association between neurons but takes into account the state
(here, the activity state) of other neurons. If all other neurons are
considered, the partial correlation is known as the inverse covariance. By
conditioning on them (and under a few assumptions), inverse correlation only
detects direct association between neuron pairs and filter out indirect
relationships that may also be observed in data \citep{sutera2014simple}. The
interest of partial correlation as an association measure has already been
shown for the inference of gene regulatory networks \citep{de2004discovery,Schafer:2005}.

Nevertheless, in particular for the connectome inference from fluorescent
data, a pre-processing is required before applying either of both measures.
Such processes (i) abstract the activity signal from fluorescent data; and
(ii) highlight characteristics of activity signals (e.g., spikes) to make
easier the assessement of the degree of association and therefore improve the
quality of the estimates of both measures.

Accordingly, ahead of (partial) correlation, filters are applied on fluorescent data in order to (i) smoothe the signal\footnote{I assume that noise will have been explained at this stage of the paper. Otherwise, we should mention something like this :\textit{Noise is due to fluctuations independent of calcium, calcium fluctuations independent of spiking activity, calcium fluctuations in nearby tissues that have been mistakenly captured, or simply by the imaging process.}}; (ii) emphasise short spikes that are indirect indicator of neuron communication; and (iii) artificially magnify spikes that occur during normal activity because we have conjectured that when a large part of the network is firing (i.e., called a network burst \citep{stetter2012model}), the estimation of degree of assocation is meaningless \citep{sutera2014simple}.


=======
A simple yet effective solution to the problem of connectome inference in
calcium imaging data consists of two steps. \cite{sutera2014simple} First,
processing the raw signals and to detect neural peak activities while filtering
noise. Second, inferring the degree of association between neurons from partial
correlation statistics, the inverse of the correlation matrix. This statitical
measure has the property to filter indirect associations between neuron and
gives a high degree of association between directly connected neurons. This
approach is the winning methods of the Connectomics challenge.
>>>>>>> b4c65e0e20727be7cd8b419c71162382dee44334

\section{Method}
\emph{A second mini-text I ask you will converge to the Methods section. Here I need
a more detailed presentation of your method. However, it has to be precise, but
still understandable for a biologist. A bit like a recipe in a cookbook for
beginners.}

<<<<<<< HEAD
=======

>>>>>>> b4c65e0e20727be7cd8b419c71162382dee44334

\newpage
\bibliography{references}

\end{document}
