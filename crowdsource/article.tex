\documentclass[wcp]{jmlr}

\usepackage{enumerate}
\usepackage{bbold}
\usepackage{booktabs}
\usepackage{natbib}
\usepackage[utf8]{inputenc}
\usepackage[super]{nth}
\setlength{\textfloatsep}{15pt}

\usepackage{algorithm}
\usepackage{algorithmic}

\jmlryear{2014}
\jmlrworkshop{Neural Connectomics Workshop}

\title{Simple connectome inference from partial correlation statistics in calcium imaging}

  \author{\Name{Antonio Sutera},
   \Name{Arnaud Joly},
   \Name{Vincent François-Lavet}, \Email{a.sutera@ulg.ac.be}\\
   \Name{Zixiao Aaron Qiu},
   \Name{Gilles Louppe},
   \Name{Damien Ernst}\and\Name{Pierre Geurts}
    \\
   \addr Department of EE and CS \& GIGA-R, University of Li\`ege, Belgium}

\begin{document}

\section{A large variety of performing methods}

\emph{
In the Results section "A large variety of performing methods", we intend to
assemble together various layman presentations of different approaches. If you
could provide a short popular-science style description of your approach (and
specifically of what you consider their main characterizing ingredients,
INVERSE COVARIANCE for you) this would be very helpful for me.  Here is not
really an issue of being precise, but of commenting substantial elements to a
public of... ignorant biologists! What's the logic of making a certain step?
How does the method work in a nutshell, without using equations but expressed
in plain words, etc. ? Then, based on all the texts I will collect from you, I
would build the section and let you edit it to correct mistakes or add
precisions.\\}

The winning solution of the Connectomics challenge \citep{sutera2014simple}
combines two main steps: (i) a pre-processing of the raw signals to detect
neural peak activities; and (ii) the inference of the network through partial
correlation statistics. The pre-processing step is crucial to obtain optimal
performance. It combines several standard filtering functions with three main
goals: to remove the noise introduced at different levels, to account for
fluorescence low decay, and to reduce the importance of high global activity in
the network. From the filtered signals, the strength of the connection between two
neurons is then assessed using partial correlations. The partial correlation
between two random variables measures their conditional dependencies, i.e.,
their associations when the effect of all the other variables has been
removed. With respect to plain correlation, the main advantage of partial
correlation for network inference is to only give a high degree of association
to directly connected neurons and thus to filter out spurious indirect
associations. Practically, partial correlations are also easily computed for
all pairs of neurons at once from a simple inversion of the data covariance
matrix.


\section{Method}
\emph{A second mini-text I ask you will converge to the Methods section. Here I need
a more detailed presentation of your method. However, it has to be precise, but
still understandable for a biologist. A bit like a recipe in a cookbook for
beginners.\\}

Our approach consists of two main steps: (i) a pre-processing of the raw
signals with several signal processing filters; and (ii) the inference of the
network through partial correlation statistics. We describe the details of
these two steps below as well as two techniques that slightly improve
performance, averaging and edge orientation. Further motivations for all steps
are given in \citep{sutera2014simple}. A simplified method is also proposed in
the same paper with only slightly lower performance.

\paragraph{Pre-processing.}

Under the simplifying assumption that neurons are on-off units, characterised
by short periods of intense activity, or peaks, and longer periods of
inactivity, the first part of our algorithm consists in cleaning the raw
fluorescence data. This pre-processing applies successively four
filtering functions to the original signal:
\begin{itemize}
  \item The first function smooths the fluorescence signal, using one of the following
low-pass filters for filtering out high frequency noise:
\begin{align} f_1(x^t_i) &= x^{t-1}_i + x^t_i + x^{t+1}_i, \label{eq:symetric-median} \\
  f_2(x^t_i) &= 0.4 x^{t-3}_i + 0.6 x^{t-2}_i + 0.8 x^{t-1}_i + x_i^t.\label{eq:weighted-asymetric-median}\\
  f_3(x^t_i) &= x^{t-1}_i + x^{t}_i + x^{t+1}_i + x^{t+2}_i, \label{eq:asymetric-median-forward} \\
f_4(x^t_i) &=  x_i^t + x^{t+1}_i  + x^{t+2}_i + x^{t+3}_i. \label{eq:asymetric-median}
\end{align}
\item To have a signal that only has high magnitude around instances where the
spikes occur, the second filter function transforms the time-series into
its backward difference 
\begin{align} g(x^{t}_{i}) &= x^{t}_i - x^{t-1}_i.
\label{eq:high-pass-filter}
\end{align}
\item To filter out small variations in the signal obtained after applying the
function $g$, as well as to eliminate negative values, we then use the following
hard-threshold filter
\begin{align}\label{eqn:hfilter}
h(x^{t}_i) &= (x^{t}_i)^c \mathbb{1}(x^{t}_i \geq \tau) \text{ with } \tau > 0,
\end{align}
where $\tau$ is the threshold parameter, $c$ is equal to 0.9, and $\mathbb{1}$ is the indicator
function.
\item The last filtering functions is defined as:
\begin{align}
 w^*(x^{t}_i) &= {(x^{t}_i + 1 )^{\left (1 + \frac{1}{\sum_{j} x^{t}_j}\right )}}^{k(\sum_{j} x^{t}_j)}
\end{align}
where the function $k$ is a piecewise linear function optimised
separately for each filter $f_1$, $f_2$, $f_3$ and $f_4$ (see the
implementation for full details). The effect of this filtering is to
magnify the importance of spikes that occur in cases of low global
activity (measured by $\sum_j x^t_j$).
\end{itemize}
The pre-processed times-series are then obtained by the application of
the following function: $w^*\circ h \circ g \circ f_i$ (with
$i=1$, 2, 3, or 4). 
  
\paragraph{Partial correlation.}
In the second main step of our approach, partial correlation is used as a
measure of the strength of the connection between neurons. Considering each
time point of the pre-processed signals as an independent measurement
of $p$ random variables, one for each of the $p$ neurons,
the partial correlation coefficient $p_{i,j}$ between neurons $i$ and $j$ is
defined by:
\begin{equation}
p_{i,j} =
-\frac{\Sigma^{-1}_{ij}}{\sqrt{\Sigma^{-1}_{ii} \Sigma^{-1}_{jj}}}, \label{eq:inverse}
\end{equation}
where $\Sigma^{-1}$, known as the precision or concentration matrix, is the
inverse of the covariance matrix $\Sigma$ of the data. The computation of all
$p_{i,j}$ coefficients thus requires the estimation of the covariance matrix
$\Sigma$ and then computing its inverse. Given that typically we have more
samples (i.e., time points) than neurons, the covariance matrix can be inverted
in a straightforward way. We nevertheless obtained some improvement by
replacing the exact inverse with an approximation using only the $M$ first
principal components, with $M=0.8p$ \citep{bishop2006pattern}.

\paragraph{Averaging.}
The performance of our method is sensitive to the values of the pre-processing
parameters: the parameter $\tau$ of the hard-threshold filter $h$ (see Equation
\ref{eqn:hfilter}), and the choice of the low-pass filter (among $\{f_1,
f_2, f_3, f_4\}$). To improve robustness, we average partial correlation
statistics obtained for all pairs $(\tau,\mbox{low-pass filter}) \in
\{0.100,0.101,\ldots,0.209\}\times \{f_1, f_2, f_3, f_4\}$. In this averaging,
functions $f_1$, $f_2$, $f_3$, and $f_4$ were given weights respectively equal
to $0.383$, $0.345$, $0.004$, and $0.268$.

\paragraph{Edge orientation.}
Partial correlation is a symmetric measure that can not be used to predict edge
orientation. A post-processing heuristic was proposed to derive an asymmetric
association score from the partial correlation coefficient. This heuristic is
based on the observation that if there is directed edges from $i$ to $j$ then
the activation of $i$ should be followed by the activation of $j$ after a
slight delay. For every pairs $(i,j)$, we computed the following term:
\begin{align}
s_{i,j} = \sum_{t=1}^{T - 1} \mathbb{1}((x_j^{t+1} - x_i^t) \in \left[\phi_1, \phi_2\right])
\end{align}
where $\phi_1$ and $\phi_2$ are parameters whose values have been
optimized through experiments to $0.2$ and $0.5$, respectively. The
final association score for an edge from $i$ to $j$ was
then the following:
\begin{align}
q_{i,j} = weight * p_{i,j} + (1-weight) * (s_{i,j}-s_{j,i}),
\label{eqn:qij}
\end{align}
with $p_{i,j}$ the average partial correlation coefficient and the parameter
$weight$ optimized close to 1 ($weight=0.997$). 
\newpage
\bibliography{references}

\end{document}
