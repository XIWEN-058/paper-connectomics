 %\documentclass[wcp,gray]{jmlr} % test grayscale version
\documentclass[wcp]{jmlr}

 % The following packages will be automatically loaded:
 % amsmath, amssymb, natbib, graphicx, url, algorithm2e

 %\usepackage{rotating}% for sideways figures and tables
\usepackage{longtable}% for long tables

 % The booktabs package is used by this sample document
 % (it provides \toprule, \midrule and \bottomrule).
 % Remove the next line if you don't require it.
\usepackage{booktabs}
 % The siunitx package is used by this sample document
 % to align numbers in a column by their decimal point.
 % Remove the next line if you don't require it.
\usepackage[load-configurations=version-1]{siunitx} % newer version
 %\usepackage{siunitx}
\usepackage[utf8]{inputenc}

 % The following command is just for this sample document:
\newcommand{\cs}[1]{\texttt{\char`\\#1}}

% % Define an unnumbered theorem just for this sample document:
%\theorembodyfont{\upshape}
%\theoremheaderfont{\scshape}
%\theorempostheader{:}
%\theoremsep{\newline}
%\newtheorem*{note}{Note}

 % change the arguments, as appropriate, in the following:
\jmlrvolume{1}
\jmlryear{2010}
\jmlrworkshop{Neural Connectomics Workshop}

\title[Connectomics challenge]{Inferring neural networks from fluorescent
                               calcium imaging using partial correlation and
                               sample weighting}

 % Use \Name{Author Name} to specify the name.
 % If the surname contains spaces, enclose the surname
 % in braces, e.g. \Name{John {Smith Jones}} similarly
 % if the name has a "von" part, e.g \Name{Jane {de Winter}}.
 % If the first letter in the forenames is a diacritic
 % enclose the diacritic in braces, e.g. \Name{{\'E}louise Smith}

 % Two authors with the same address
%  \author{\Name{Author Name1\nametag{\thanks{with a note}}} \Email{abc@sample.com}\and
%   \Name{Author Name2} \Email{xyz@sample.com}\\
%   \addr Address}

 % Three or more authors with the same address:
 \author{}


 % Authors with different addresses:
 % \author{\Name{Author Name1} \Email{abc@sample.com}\\
 % \addr Address 1
 % \AND
 % \Name{Author Name2} \Email{xyz@sample.com}\\
 % \addr Address 2
 %}

\editor{Editor's name}
 % \editors{List of editors' names}

\begin{document}

\maketitle

\begin{abstract}
This is the abstract for this article.
\end{abstract}

\begin{keywords}
List of keywords
\end{keywords}


\section{Introduction}\label{sec:intro}

% BG : brain, complex organ
Understanding how the brain works is a key to understand and to treat
brain pathologies and disorders such as Parkinson's disease, epilepsy.
Retrieving the connectomes, neurons connectivity map, will shed new lights on
the anatomical and functional connectivity of the brain.
The human brain is a complex biological organ, which is formed by 100
billions of neurons with 7000 synaptic connections on average. Inferring the
neuron connectivity network through dissection would be an impossible
and daunting task.

% Issue : we have data, but need to have to infer the connectome
% TODO Shed light on the difficulty of the inverse problem:
% noise, time scale (optical imaging speed vs neuron speed), decay of
% fluorescent signal.
Cutting edge optical devices (TODO CITE) allows to track simultaneously
the neural activity of thousands of neurons through the fluorescent
calcium indicator. The neuron connectivity network could be inferred
from those neuron activity signals.

% Solution
In this paper, we describe a simple and a theoretically grounded approach
based on partial correlation to infer the neural connectivity network.
This is also the winning method of the Connectomics challenge.


\section{Method}

% Introduce notation + goal in mathematical score
% TODO check convention for y_{ij}
Given $X \in \mathcal{R}^{n \times p}$  a set of $p$ time series of fluorescent
calcium concentration of size $n$ from $p$ neurons, the goal is to infer the
connectivity network $y \in \left\{0, 1\right\}^{p \times p}$ where
non zero values indicate that the neuron $i$ have an outward connection to
the neuron $j$.



% Introduce partial correlation
% Introduce directivity
% Introduce filtering

\section{Empirical experiments}
\subsection{Datasets}
\subsection{Metrics}
\subsection{Data filtering}
\subsection{Signal distribution}
\subsection{ROC AUC results}
\subsection{AUPRC}

\section{Conclusion}



\bibliography{references}


\end{document}
